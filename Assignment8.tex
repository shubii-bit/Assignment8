\documentclass{article}
\usepackage[utf8]{inputenc}
\usepackage[a4paper,top=3cm,bottom=2cm,left=3cm,right=3cm,marginparwidth=1.75cm]{geometry}

%% Useful packages
\usepackage{amsmath,amsthm,amssymb,amsfonts}
\usepackage{graphicx}
\usepackage{listings}
\usepackage[colorlinks=true, allcolors=blue]{hyperref}
\usepackage{xcolor}
\usepackage[normalem]{ulem}
\useunder{\uline}{\ul}{}
\usepackage{longtable}

\definecolor{codegreen}{rgb}{0,0.6,0}
\definecolor{codegray}{rgb}{0.5,0.5,0.5}
\definecolor{codepurple}{rgb}{0.58,0,0.82}
\definecolor{backcolour}{rgb}{0.95,0.95,0.92}

\lstdefinestyle{mystyle}{
    backgroundcolor=\color{backcolour},   
    commentstyle=\color{codegreen},
    keywordstyle=\color{magenta},
    numberstyle=\tiny\color{codegray},
    stringstyle=\color{codepurple},
    basicstyle=\ttfamily\normalsize,
    breakatwhitespace=false,         
    breaklines=true,                 
    captionpos=b,                    
    keepspaces=true,                 
    numbers=left,                    
    numbersep=5pt,                  
    showspaces=false,                
    showstringspaces=false,
    showtabs=false,                  
    tabsize=2
}

\lstset{style=mystyle}


\title{MICROPROCESSOR LAB}
\author{Shubham Shrivastava}
\date{January 2021}

\begin{document}
\maketitle

\section*{Question 1}
Construct the triangles in table:\\
\begin{figure}[h!]
    \centering
    \includegraphics{ass8p.JPG}
    \caption{Given}
    \label{fig:my_label}
\end{figure}
\section{Solution}
(ii)This traingle can be constructed in following way\\
$\\$
Steps of construction:\\ 
(i)Draw a line segment $PR$ of length 4.7 cm where $P$ is at (0,0)\\
$\\$
(ii)Now, we draw a line from P at an angle of 30$^\circ$  with $PR$, the line would have the equation as
\begin{gather}
     y = \tan(30^\circ)x
\end{gather}
(iii) Drawing another line from R having an equation
\begin{equation}
    y=-tan60^\circ
\end{equation}
from (i) and (ii), on solving\\
\(\frac{x}{\sqrt{3}}\)=-\sqrt{3}(x-4.7)\\
x=-3x + 3 * 4.7\\
4x = 14.7\\
x=3.67\\
putting the value of x in (i), we get\\
y=\(\frac{3.67}{1.732}\)\\
y=2.12\\
Thus Joining \textbf{P(3.67,2.12),Q(0,0),R(4.7,0)} we would obtain the required triangle
\newpage
\begin{figure}[h!]
    \centering
    \includegraphics{xyz.JPG}
    \caption{Python}
    \label{fig:my_label}
\end{figure}
\newpage
\section*{Solution for (vi)}
Given, PQ=2.5cm, QR=4cm, PR=3.5cm\\
\begin{equation}
    p = \frac{a^{2} + c^{2}-b^{2}}{2*a}
\end{equation}
\begin{equation}
     q = \sqrt(c^{2}-p^{2})
\end{equation}
\begin{figure}[h!]
    \centering
    \includegraphics{bcd.JPG}
    \caption{Python}
    \label{fig:my_label}
\end{figure}




\newpage
\section*{Explaination for other subparts}
$=>$(i)In subpart (i) and (iv) We know that \\
sum of angles of a triangle = $180^\circ$
$\\$\\
But here, it doesn't seem so\\
Since sum of two angles cannot be greater than $180^\circ$\\
therefore, $\Delta$ is not possible.\\
$\\$
$=>$(iii)It will be constructed using the same method as we did for subpart (ii)($\Delta$ PQR)\\
$\\$
$=>$subparts (v), (vii),(viii)  will be constructed using the same method as we did for subpart (vi)\\


\section*{Question 2 }
Construct a quadrilateral ABCD such that
BC = 4.5, AC = 5.5,CD = 5, BD = 7 and
AD = 5.5\\
\section*{Solution}
Steps of constructions:\\
Firstly, we will draw a line AC of 5.5 cm where A is at (0,0)\\
then, taking A as center with the radius of 5.5 cm we get a circle whose equation is\\
\begin{equation}
    (x)^{2}+(y)^{2}=(5.5)^{2}\\
\end{equation}
similarly,by taking C as centre,we get\\
\begin{equation}
    (x-5.5)^{2}+(y-0)^{2}=(5)^{2}\\
\end{equation}
on solving these two equations, we get a point of intersection which is basically D of our quadrilateral\\
\begin{gather*}
    (x-5.5)^{2}+(y-0)^{2}=(5)^{2}\\
    x^{2}+(5.5)^2-11x + y^{2}=25\\
    (5.5)^2-y^{2}+(5.5)^2-11x+y^{2}=25\\
    x=3.22\\
    \mbox{putting the value of x in 1}\\
    y^{2}=(5.5)^{2}-(3.2)^{2}\\
    y=4.45\\
\end{gather*}
mark this point as D\\
Now, join AD and CD\\
then, we further proceed with taking two equations of circle by taking C and D as centers resp.\\
\begin{equation}
    (x-3.22)^{2}+(y-4.45)^{2}=(7)^{2}\\
\end{equation}
\begin{equation}
    (x-5.5)^{2}+(y-0)^{2}=(4.5)^{2}\\
\end{equation}
on solving, we get\\
x=1.68 & y=-2.37\\
Mark this point as B and join AB and BC and we would obtain required quadrilateral\\
\begin{figure}[h!]
    \centering
    \includegraphics[height=400pts, width=500 pts]{im8.JPG}
    \caption{Figure generated using python}
    \label{fig:my_label}
\end{figure}


\end{document}